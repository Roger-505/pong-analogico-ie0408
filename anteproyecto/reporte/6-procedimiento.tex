\section*{Procedimiento}

Este procedimiento describe los pasos necesarios para la implementación del circuito, y se omite la última etapa que incluye el altavoz (efectos de sonido).

\begin{enumerate}
\item \textbf{Preparación del Área de Trabajo y Componentes}
\begin{itemize}
\item Reúne todos los componentes listados en la sección de \textit{Lista de componentes} del documento, incluyendo amplificadores operacionales, flip-flops, comparadores, resistencias, capacitores, potenciómetros y el multiplexor analógico CD4066B.
\item Verifica el funcionamiento del osciloscopio con modo XY, asegurándote de que esté correctamente calibrado.
\item Organiza el equipo de medición como fuentes de alimentación DC, multímetros y generadores de señales si es necesario.
\end{itemize}

\item \textbf{Construcción del Oscilador en Cuadratura}
\begin{itemize}
\item Ensambla el circuito oscilador en cuadratura según el diseño proporcionado. Este circuito debe generar dos señales sinusoidales desfasadas 90$^{\circ}$ ( y ).
\item Asegúrate de que las condiciones de Barkhausen se cumplan: \( R_1 C_1 = R_2 C_2 = R_3 C_3 \quad \text{y} \quad A \cdot \beta = 1 \)



\item Verifica las salidas en el osciloscopio: una señal debería aparecer en el eje X y la otra en el eje Y, generando un círculo perfecto en el modo XY del osciloscopio.
\end{itemize}

\item \textbf{Implementación de los Circuitos Desfasadores de 360$^{\circ}$ y Comparadores Ajustables}
\begin{itemize}
\item Conecta los desfasadores de 360$^{\circ}$ utilizando potenciómetros de doble gang para ajustar la posición de las paletas A y B. Este circuito permitirá modificar el ángulo de las paletas en la pantalla.
\item Posteriormente, conecta los comparadores ajustables que generarán ondas cuadradas (A${\mathrm{on}}$ y B${\mathrm{on}}$) para controlar el tamaño de las paletas.
\item Verifica que al ajustar los potenciómetros, las paletas cambien de tamaño y posición de manera simétrica.
\end{itemize}

\item \textbf{Montaje del Sistema de Control de Dibujo y Multiplexado}
\begin{itemize}
\item Ensambla el circuito comparador con flip-flop que genera la señal de reloj (CLK) y su inversa ($\overline{\text{CLK}}$), necesarias para alternar entre el dibujo de la pelota y las paletas.
\item Implementa el multiplexor analógico 2 a 1 utilizando el CD4066B. Este circuito seleccionará si se dibujan las paletas o la pelota en función de la señal .
\item Configura las señales de control PL para gestionar los rebotes de la pelota contra las paletas.
\end{itemize}

\item \textbf{Control del Movimiento de la Pelota}
\begin{itemize}
\item Conecta el circuito flip-flop y el circuito RC para controlar la dinámica de la pelota. Este circuito regula la velocidad de la pelota y su dirección, ajustada mediante potenciómetros que controlan la constante de tiempo .
\item Implementa los comparadores de ventana para detectar colisiones entre la pelota y las paletas. Verifica que la señal \textit{BOUNCE} se active correctamente en caso de colisión.
\end{itemize}

\item \textbf{Detección de Límites y Control de Direcciones}
\begin{itemize}
\item Ensambla los comparadores de ventana que detectan los límites horizontales y verticales de la pantalla. Estas señales determinarán si la pelota debe rebotar al llegar al borde.
\item Integra el circuito de control de dirección que invertirá las señales X${\mathrm{rev}}$ y Y${\mathrm{rev}}$, cambiando la trayectoria de la pelota en función de las colisiones detectadas.
\end{itemize}

\item \textbf{Control de Turnos y Señalización con LEDs}
\begin{itemize}
\item Conecta el circuito flip-flop para gestionar los turnos de los jugadores. La señal \textit{TURN} alternará entre los jugadores cada vez que la pelota rebote en una paleta.
\item Añade LEDs indicadores para mostrar de quién es el turno actual, asegurando que un LED se apague y otro se encienda tras cada rebote.
\end{itemize}

\item \textbf{Pruebas Finales y Ajustes}
\begin{itemize}
\item Conecta todas las etapas del circuito y verifica el funcionamiento global en el osciloscopio. Asegúrate de que las paletas y la pelota se dibujen correctamente y que los controles de posición, tamaño y velocidad funcionen según lo esperado.
\item Ajusta las constantes de tiempo y las configuraciones de los comparadores para optimizar el rendimiento del circuito.
\end{itemize}
\end{enumerate}


